% Contains a list of standard packages and command declarations.

%-------------------------------------------------------------------------------
% SMALLER MARGINS
%-------------------------------------------------------------------------------
\usepackage{xcolor} 
\usepackage{wrapfig}
\newcommand{\red}[1]{\textcolor{red}{(#1)}}
\newcommand{\blue}[1]{\textcolor{blue}{(#1)}}
\usepackage{natbib}
\usepackage{soul}
\setstcolor{red}

\usepackage{amsmath}
\usepackage{amsthm}
\usepackage{amssymb}
\usepackage{bm} % for bold math symbols
\usepackage{bbm} % for bold fonts in math mode
\usepackage{mathtools}
%\usepackage[ruled,linesnumbered]{algorithm2e}
% \usepackage{algorithm}
\usepackage{enumitem}
\usepackage{longtable}
\usepackage[ruled, vlined,linesnumbered]{algorithm2e}
\usepackage{hyperref}
% Set 2 or 1.5 to increase line spacing.
% \linespread{2}

% PARAGRAPH INDENTATION 
% \newlength\tindent
% \setlength{\tindent}{\parindent}
% % \setlength{\parindent}{0pt} % set paragraph indentation to zero
% \renewcommand{\indent}{\hspace*{\tindent}} % use \indent to manually indent a paragraph with the original \parindent length


% \usepackage[font=small,labelfont=bf,tableposition=top]{caption}
% \usepackage[font=footnotesize]{subfig}

\usepackage{subcaption}
\usepackage{diagbox}
%--------------------------------------------------------------------------------
% For Multi-line cells in tables
\usepackage{makecell}
\setlength\extrarowheight{5pt}
% \usepackage[thinlines]{easytable}

%--------------------------------------------------------------------------------

% REDUCE WHITESPACE AFTER FIGURE
% \setlength{\belowcaptionskip}{-12pt}

% REMOVE WHITESPACE BEFORE AND AFTER FLOAT 
\setlength{\textfloatsep}{0.5cm}
\setlength{\floatsep}{0.5cm}

% REMOVE SPACE BEFORE AND AFTER EQUATIONS 
\makeatletter
\g@addto@macro\normalsize{%
\setlength\abovedisplayskip{6pt} %default 6
\setlength\belowdisplayskip{6pt}
\setlength\abovedisplayshortskip{6pt} %default 6
\setlength\belowdisplayshortskip{6pt}
}
\makeatother





% % EULER CALIGRAPHIC SCRIPT FONT
% \usepackage[mathscr]{euscript} 

% Define the Theorems and Definitions

\newtheorem{theorem}{Theorem}
\newtheorem{lemma}{Lemma}
\newtheorem{proposition}{Proposition}
\newtheorem{claim}{Claim}
\newtheorem{corollary}{Corollary}
\newtheorem{assumption}{Assumption}


\theoremstyle{definition}
\newtheorem{definition}{Definition}
\newtheorem{remark}{Remark}

% Define new commands
\newcommand{\lp}{\left (} % Left parantheses
\newcommand{\rp}{\right )} % Right parantheses
\newcommand{\lb}{\left [}
\newcommand{\rb}{\right ]}

\newcommand{\mc}[1]{\mathcal{#1}}
\newcommand{\mbb}[1]{\mathbb{#1}}
\newcommand{\mf}[1]{\mathfrak{#1}}
\newcommand{\indi}[1]{\mathbbm{1}_{ \{#1\} }}

\newcommand{\reals}{\mathbb{R}} 




\newcommand{\X}{\mathcal{X}}
\newcommand{\Y}{\mathcal{Y}}

\DeclareMathOperator*{\argmax}{arg\,max}
\DeclareMathOperator*{\argmin}{arg\,min}


\newcommand{\tbf}[1]{\textbf{#1}}



% Commands for pushing text to the left and right in an align environment
\makeatletter
\newcommand{\pushright}[1]{\ifmeasuring@#1\else\omit\hfill$\displaystyle#1$\fi\ignorespaces}
\newcommand{\pushleft}[1]{\ifmeasuring@#1\else\omit$\displaystyle#1$\hfill\fi\ignorespaces}
\makeatother


% ARGMIN AND ARGMAX COMMANDS
% \DeclareMathOperator*{\argmax}{arg\,max}
% \DeclareMathOperator*{\argmin}{arg\,min}
